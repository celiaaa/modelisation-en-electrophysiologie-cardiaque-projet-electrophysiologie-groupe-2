\documentclass[a4paper,12pt,twoside]{report}

\usepackage[french]{babel}  % Pour le français
\usepackage[utf8]{inputenc} % Pour taper les caractères accentués
\usepackage[T1]{fontenc} %Not needed on school computers

\usepackage{amsmath}  % Ces trois paquets donnent accès à 
\usepackage{amsfonts} % des symboles et formulations 
\usepackage{amstext}  % mathématiques
\usepackage{hyperref} % Permet de faire automatiquement des liens dans les
                      % documents
\usepackage{graphicx} % Permet d'insérer des images
\DeclareGraphicsExtensions{.png} % Mettre ici la liste des extensions des
                                 % fichiers images

% On peut choisir la police en utilisant un paquet 
%\usepackage{newcent}
\usepackage{lmodern}
%\usepackage{cmbright} % Computer Modern Bright

% Une des nombreuses manière de modifier les marges par défaut
\usepackage{geometry}
\geometry{vmargin=2cm,hmargin=2.5cm,nohead}

% If si units are needed : \usepackage{siunitx}

% On peut redéfinir certaines longueurs, par exemple l'espacement entre les
% paragraphes:
\setlength{\parskip}{0.25cm}

% Quelques définitions 
\def \rr {{\mathbb R}} % L'ensemble R
\def \cc {{\mathbb C}} % L'ensemble C
\def \nn {{\mathbb N}} % L'ensemble N
\def \zz {{\mathbb Z}} % L'ensemble Z

% Les informations de la page de titre (page de titre séparée pour un 'report').
\title{
\vspace*{1in}
\textbf{Electrocardiogramme}}

\author{Rémi HERNANDEZ\\
        Romuald MAKOSSO NOMBO\\
        Amandine CAUT\\
        Célia BOULTABI\\
		\vspace*{0.5in} \\		
		Tuteur : Y.Courdiere \\	
		\vspace*{0.5in} \\		
		Université de Bordeaux\\		
       } 
%--------------------Make usable space all of page
\setlength{\oddsidemargin}{0in} \setlength{\evensidemargin}{0in}
\setlength{\topmargin}{0in}     \setlength{\headsep}{-.25in}
\setlength{\textwidth}{6.5in}   \setlength{\textheight}{8.5in}
%--------------------Indention
\setlength{\parindent}{1cm}


% \thanks: permet de mettre une note de bas de page pour l'auteur
% \href: insère un lien, ici vers l'application 'mailto'
% \tt: police monospace
\date{\today} % \today pour la date courante 

\cleardoublepage

\begin{document}

\maketitle % Page de titre automatique à partir des infos ci-dessus

% La commande cleardoublepage est utilisée pour s'assurer que la page suivante
% est une page de droite lorque l'on imprime recto-verso.



\pagebreak
\hspace{0pt}

\vfill
\hspace{0pt}
\pagebreak




\cleardoublepage
\tableofcontents % Table des matière automatique à partir des chapitres,
                 % sections, etc du document



\chapter{Étude théorique} 

L'électrophysiologie permet de modéliser la propagation de différentes ondes dans le cœur. \\
Pour cela nous allons utiliser cette équation:\\

$$\partial_t u(x,t) - \partial_x(a(x) \partial u(x,t)) = f(t,x)$$

Afin de résoudre cette équation il faut certaines conditions initiales et aux bords. Nos recherches dans différents livres et articles nous permettent alors de déduire les conditions suivantes: \\
\\
$$ 
\left\{ 
	\begin{array}{ll}
		\partial_t u(x,t) - \partial_x(a(x) \partial u(x,t)) = f(t,x) \\
		\partial_x u(x,t) = 0 \ sur \  \partial \Omega\\
		u(0,x) = u_0(x) \ t \in [0,T]
		
	\end{array}
\right. 
$$


Il est nécessaire que $a(x) > 0$, afin que le phénomène ne soit pas réversible! En effet, on ne peut inverser la diffusion de la chaleur. \\ 
On remarque par ailleurs que l'on a l'équation chaleur pour le cas $a(x) = 1$. On résoudra cette équation dans le chapitre 2.






\chapter{programmation dans un cas simplifié} 

2) Situation où l'on peut calculer des solutions analytiques: \\
dans le cas ($a(x) = 1$) on a l'équation de la chaleur. De plus avec nos conditions aux bords, ie les dérivées sont nulles, on a deux cas: \\
 


cas 1:\\ 
$f(x,t) = 0$ \\
la solution est la fonction u définie par : \\
$$ u(x,t) =  cos(k \pi x) exp(-k^2 \pi^2 t) $$ 

cas 2:\\
$f(x,t) \neq 0$ \\
Ici, f(x,t) est à valeurs dans un espace de Hilbert. $ f \in L^2 $. \\
u sera de la forme: \\
$$ u(x,t) = \int_{- \infty}^{+ \infty} E(x-y,t)u_0(y)dy + \int_{-\infty}^{+\infty} \int_{0}^{t} E(x-y,t-s)f(y,s)dyds$$

$$ 
E(x,t) = \left\{ 
	\begin{array}{ll}
		\frac{1}{\sqrt{4 \pi t}} exp(\frac{-x^2}{4t}) \mbox{ si } t>0 \\
		 0 si t \leq 0  
	\end{array}
\right. 
$$

\section{Schémas numériques pour la résolution}

Pour la résolution de cette équation, on utilise un maillage uniforme avec un pas de $\Delta t$ en temps et $\Delta x$ en espace. \\
La dérivée seconde en espace est approchée par : 
\begin{equation}
\frac{\partial ^2u}{\partial ^2x} = \frac{u(x_{i+1},t_j) - 2u(x_i,t_j) + u(x_{i-1},t_j)}{\Delta x^2}
\end{equation}

\subsection{Euler explicite}

On approche $\frac{\partial u}{\partial t}(x_i,t_j)$ par :
\begin{equation}
\frac{u(x_i,t_j+1) - u(x_i,t_j)}{\Delta t}
\end{equation}
On obtient le schéma suivant :\\
\begin{equation}
\frac{u(x_i,t_j+1) - u(x_i,t_j)}{\Delta t} - \frac{u(x_{i+1},t_j) - 2u(x_i,t_j) + u(x_{i-1},t_j)}{\Delta x^2} = f(x_i,t_j)
\end{equation}
Pour la suite on écrira $u^j_i$ à la place de $u(x_i,t_j)$. \\ 
Aussi on utilisera la notation :
$U^j = \left( \begin{array}{c}
u^j_1\\
u^j_2 \\
\vdots \\
u^j_N
\end{array} \right)$

Le schéma s'écrit sous forme matricielle :
\begin{equation}
\frac{U^{j+1}-U^j}{\Delta t} + A_{\Delta x}U^j = F^j
\end{equation}
Avec :
\begin{equation}
A_{\Delta x} = \frac{1}{\Delta x^2}
\begin{pmatrix}
   2 & -1 & 0 & \ldots & \ldots & 0\\
   -1 & 2 & -1 & 0 \ddots & & \vdots \\
   0 & \ddots & \ddots & \ddots & \ddots & \vdots\\
   \vdots & \ddots & \ddots & \ddots &\ddots & 0\\
   \vdots & & \ddots & -1 & 2 & -1\\
   0 & \cdots & \cdots & 0 & -1 & 2
\end{pmatrix}
\end{equation}
Et on a $F^j = \left( \begin{array}{c}
f(x_1,t_j)\\
f(x_2,t_j) \\
\vdots \\
f(x_N,t_j)
\end{array} \right)$




\end{document}